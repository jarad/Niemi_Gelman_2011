\documentclass[12pt]{article}
\usepackage[top = .25in, bottom = .25in, left = .25in, right = .25in]{geometry}

\usepackage{graphicx, hyperref}

\title{Statistical graphics:\\making information clear---and beautiful\footnote{For {\em Significance} magazine.  We thank the National Science Foundation, National Security Agency, and Institute of Education Sciences for partial support of this work.}}
\author{Jarad Niemi\footnote{Department of Statistics \& Applied Probability, University of California, Santa Barbara} and Andrew Gelman\footnote{Department of Statistics and Department of Political Science, Columbia University}}
\date{7 April 2011}

\begin{document}

\maketitle

In the toolkit section of the March issue of Significance Magazine, Julian Champkin looked at four examples of what we would term Information Visualization (InfoVis). This issue we will discuss statistical graphics where the importance has shifted from making the graphic beautiful to making it clear. In keeping with the theme of this issue, we will look at laboratory confirmed cases for the city of Harare, Zimbabwe in the face of a measles outbreak that initiated in fall 2009 and continued throughout 2010. 

Before the outbreak, measles vaccination coverage in Zimbabwe had reached 92\%. In early fall of 2009, cases were reported in a number of districts in Zimbabwe. Immediately public health officials responded to these cases by collecting samples for diagnostic testing and vaccinating children in affected villages. By late May 2010, there were 7,754 suspected cases, 508 laboratory confirmed cases (61 of Zimbabwe's 62 districts having at least one confirmed case), and 517 deaths. The vast majority of these cases had not been vaccinated previously. From 24 May to 2 June, a mass vaccination campaign for the whole country was conducted vaccinating more than 5 million children. As of 12 December 2010, there were a total of 13,783 suspected cases, 693 confirmed cases, and 631 deaths.  With no confirmed cases in the following months, the outbreak is assumed to have ended.\footnote{Our data come from World Health Organization's Zimbabwe epidemiological bulletins.}

\subsubsection*{Default graphs in Excel and R}

\begin{figure}[ht]
\centering
\includegraphics[scale=1]{ExcelDefaultFigure}
\caption{Default creation of the number of confirmed cases in Harare, Zimbabwe during the measles outbreak that began in November of 2009.}
\label{fig:ExcelDefaultFigure}
\end{figure}

\begin{figure}[ht]
\centering
\includegraphics[scale=0.6]{RDefaultFigure}
\caption{Default creation of the number of confirmed cases in Harare, Zimbabwe during the measles outbreak that began in November of 2009.}
\label{fig:RDefaultFigure}
\end{figure}

Figures \ref{fig:ExcelDefaultFigure} and \ref{fig:RDefaultFigure} provide default plots of the time series of cumulative confirmed measles cases, first in Excel and then in R.  The graphs provide the basic outbreak information but not in a complete, concise, or visually appealing way. In the Excel version, the $x$-axis is too busy---the reader must spend a fair effort to decode the labels and figure out the time scale involved---the $y$-axis is unlabeled, the default action adds a legend that has an uninformative title and is unnecessary when only one series exists. On the positive side, creating a graphic in Excel requires the user to choose a chart type; our choice was a `marked line plot' and the default blue color is visually appealing. In contrast, the default graph in R produces a terrible point chart with only one x-axis tick demarcating the start of 2010 and only slightly helpful labels. 

The Excel and R graphs share an even more serious problem, which is that they are labeled and scaled to fill an entire screen.  The time series presented here is uncomplicated and could easily fit on a small portion of the page, if the labeling were sized appropriately.

\subsubsection*{Constructing a more beautiful and informative summary of data and inferences}

Our purpose in showing these examples is not to denigrate Excel or R, but rather to point out that using defaults to produce figures, while useful for exploratory data analysis, does not produce production quality figures.  When producing quality figures, every decisions needs to be made consciously and with intent. Improving figures requires two key decisions:
\begin{itemize}
\item Who is your target audience?
\item What are you trying to show?
\end{itemize}
Once these decisions have been made, they will guide your figure creation choices.

In the Harare measles outbreak example, we are interested in showing the timing of the WHO's mass vaccination campaign relative to the outbreak in order to provide public health officials with a retrospective view of the policy decision. Since public health officials are accustomed to visualizing outbreaks as the number of infected individuals at a given time, this suggests plotting the number of new confirmed cases rather than the cumulative cases. In addition, we can subdivide the public health officials into those who are interested in this particular outbreak and those who are interested in lessons learned from this outbreak that will inform future policy. The former would likely be interested in the actual dates of this outbreak while the latter would like be more interested in the number of weeks since disease discovery. Finally, although understanding the total number of \emph{confirmed} measles cases is interesting, a more relevant quantity is the \emph{estimated} total cases and its associated uncertainty.

A few hours of trial and error in R yielded Figure \ref{fig:RImprovedFigure}, our improved version of the confirmed cases data, along with estimates and uncertainties for total cases, and mass vaccination campaign dates.
\begin{figure}[ht]
\centering
\includegraphics[scale=0.6]{RImprovedFigure}
\caption{Improved figure displaying the number of confirmed cases (black) and estimated total cases (red) in Harare, Zimbabwe during the measles outbreak that began in November of 2009 and the mass vaccination campaign (green) that was held from 24 May to 2 June.}
\label{fig:RImprovedFigure}
\end{figure}
The figure seamlessly incorporates both confirmed and estimated total cases with uncertainty by adding an additional $y$-axis for the estimated cases. The mass vaccination campaign is indicated with a solid green box, labeled with the time-since-detection to address the different needs of public health officials, some of whom will be interested in exact dates and others who care about the general time pattern.

Initially, we thought the message of this plot would be that public health officials vaccinated too late as the main peak of the outbreak occurred much earlier, but after creating the figure we were surprised to see the second peak shortly after the vaccination campaign. Perhaps this is due to the associated public awareness campaign and more people reporting measles cases, but the three data points immediately after the campaign belie that hypothesis. So perhaps a second peak was on its way, but was stymied by the vaccination campaign. In fairness to public health officials, the campaign included all of Zimbabwe, and Harare was one of the earliest outbreak locations. 

Most of this development was accomplished by trying to \emph{put ourselves in the shoes} of public health officials and determine what might be important to them, but we used some guiding principles along the way:
\begin{itemize}
\item Avoid distracting elements
\item Use informative color to visually associate elements
\item Keep the figure simple (and therefore interpretable).
\end{itemize}
As an example of avoiding distracting elements, we had initially used a marked line plot, similar to the Excel plot earlier, for the new confirmed cases data, but the resulting lines had a sawtooth pattern that dominated the figure without providing meaningful information. We used red, a color representing emergency or danger, to combine all the figure components related to estimated cases, and green, a color representing earth and health, to combine the elements associated with the vaccination campaign.

We think Figure \ref{fig:RImprovedFigure} is a much improved data summary, but we are under no illusion that it is perfect.  In particular, we are not happy with the potentially confusing double $y$-axis and we suspect there is a clearer way to simultaneously display confirmed counts on one scale and estimated totals on the other.  At the level of process, the graph was awkward to create.  We made countless tweaks to our R script and hard-coded various details such as the positions of the labels, the scaling of the axes, the long tick mark dividing 2009 from 2010, and a slight shifting of the zero points so they would be clear of the $x$-axis (which we wanted to be precisely at zero---contrary to the R default---so make it visually apparent that the estimated rate declines to zero at the end of the time series).  Some manual adjustments may always be necessary to prepare a complex presentation-quality graph, but R (as currently configured) is not the ideal environment for this process.

\subsubsection*{Further graphs for data exploration}

Given our curiosity about the timing of the vaccination campaign relative to the outbreak, it is tempting to put the data for multiple cities into Figure  \ref{fig:RImprovedFigure}, but we decided this would unnecessarily complicate the figure.

Instead, the concept of \emph{small multiples} can be used to compare similar plots as shown in Figure \ref{fig:smallMultiples}. This figure shows the original Harare data along with that from Bulawayo (the second largest city in Zimbabwe), and Mashonaland (a region in northeastern Zimbabwe). We have chosen to align the plots vertically to emphasize the relative outbreak peaks. From this figure it is clear that Harare and Bulawayo had outbreak peaks at approximately the same time while Mashonaland's peak was later. This suggests the outbreak moved out from cities to individuals in Mashonaland. If emphasizing relative peak heights was of interest, a horizontal series of plots would have been more useful.
\begin{figure}[ht]
\centering
\includegraphics[scale=0.4]{smallMultiplesSimple}
\caption{Figure displaying the concept of small multiples in order to compare outbreak profiles in Harare, Bulawayo, and Mashonaland with the number of confirmed cases (black) and estimated total cases (red)  during the measles outbreak that began in November of 2009 and the mass vaccination campaign (green) that was held from 24 May to 2 June.}
\label{fig:smallMultiples}
\end{figure}

In order to create these small multiple plots, we utilized another set of guiding principles:
\begin{itemize}
\item Keep the x and y axes on the same scale
\item Eliminate repetitive information
\item Maintain consistency across plots.
\end{itemize}
The scales for the three axes are repeated and so the figures are immediately comparable for outbreak timing and intensity.  For example, Bulawayo had many fewer cases than Harare.  For these graphs we decided to remove some of the detail in the axes, as they would unnecessarily complicate the small multiples.  In addition to the $x$ and $y$ axes on the small scale, we have also maintained the outbreak estimates and 95\% intervals as well as the color scheme in order to maintain consistency across the plots.

In this way, Figures \ref{fig:RImprovedFigure} and \ref{fig:smallMultiples} work together as a story, and in a way contrary to the usual practice of statisticians.  We are generally taught to start with exploratory plots and then move toward presentation-quality graphs.  But in this case the visually attractive overview sets the stage for more focused data exploration.

We realize these figures have not been as eye-catching as the InfoVis figures in the March issue of Significance Magazine, but we hope they illustrate how graphical display of data can be a useful tool in understanding the measles outbreak in Zimbabwe.  We envision a future with multilayer interactive graphics. The public layer would be an InfoVis figure to catch the reader's attention. The reader could dive a layer deeper to obtain a statistical graphic such as those displayed in this article. Diving one more layer down would produce the data table from which both the statistical graphic and the InfoVis were produced. Each layer would be interactive allowing the reader to present the data in the way they felt most appropriate for their needs. In this way, we view InfoVis and statistical graphics as complementary tools for understanding data.


\end{document}